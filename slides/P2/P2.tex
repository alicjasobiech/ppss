\documentclass{beamer}
\usepackage{ulem}
\usepackage{tikz}
\usepackage{booktabs}
 \usepackage{graphicx,threeparttable,caption}
\usetikzlibrary{shapes,snakes}
\usepackage[beamer,customcolors]{hf-tikz}
\usepackage{nicematrix}
\usepackage{xcolor}
\usepackage{makecell}
\usepackage{array}
\usepackage{csquotes}
\usepackage{csquotes}
\usepackage{minted}
\captionsetup{labelformat=empty,labelsep=none}

\graphicspath{ {./png/} }
\tikzset{hl/.style={
    set fill color=red!80!black!40,
    set border color=red!80!black,
  },
}
\AtBeginSection[]{
  \begin{frame}
  \vfill
  \centering
  \begin{beamercolorbox}[sep=8pt,center,shadow=true,rounded=true]{title}
    \usebeamerfont{title}\insertsectionhead\par%
  \end{beamercolorbox}
  \vfill
  \end{frame}
}
%\usecolortheme[orchid]{structure}
\usetheme[hideothersubsections]{PaloAlto}
\makeatletter
\patchcmd{\csq@bquote@i}{{#6}}{{\emph{#6}}}{}{}
\makeatother
%\usecolortheme{orchid}
%\usefonttheme{professionalfonts}
\newcommand{\soutthick}[1]{%
   \textcolor{red}{
   \renewcommand{\ULthickness}{1pt}%
      \sout{#1}%
   \renewcommand{\ULthickness}{.4pt}% Resetting to ulem default
   }
}
\newcommand{\centered}[1]{\begin{tabular}{l} #1 \end{tabular}}
\setbeamertemplate{section in toc}[square]
\setbeamertemplate{subsection in toc}[square]
\setbeamertemplate{secion in sidebar}[shaded]
\setbeamertemplate{items}[square]
\setbeamercovered{transparent} 

\title[]{Programming in Python for Social Scientists}
\subtitle{JSON}
\author[]{Mikołaj Biesaga\\ \small{\color{blue}{\href{mailto:m.biesaga@uw.edu.pl}{m.biesaga@uw.edu.pl}}}}
\institute{\includegraphics[width = 4 cm]{uw.png}}
\date{\today}
\begin{document}
\begin{frame}
   \titlepage
\end{frame}

\begin{frame}
    \frametitle{Data Formats}
    \only<+>{
        Marianna is a 17 years old young lady. Although her main field of interest is physics (especially quantum physics and string theory), she also fancies sport. Her favorite physical activities are fishing and football.
        Marian, on the other hand, is a naughty 15 years old boy who only loves literature, especially Szymborska poems touches his heart.
    }
    \only<+>{
        \textcolor{red}{Marianna} is a \textcolor{blue}{17} years old young lady. Although her main field of interest is \underline{physics} (especially \textbf{quantum physics} and \textbf{string theory}), she also fancies \underline{sport}. Her favorite physical activities are \textbf{fishing} and \textbf{football}.
        \textcolor{red}{Marian}, on the other hand, is a naughty \textcolor{blue}{15} years old boy who only loves \underline{literature}, especially Szymborska \textbf{poems} touches his heart.
    }
    \only<+>{
        \resizebox{\textwidth}{!}{
            \begin{tabular}{l | c | c | c | c | c | c | c | c }
            Name & Sex & Age & Interest A & Interest A1 & Interest A2 & Interest B & Interest B1 & Interest B2\\
            \hline \hline
            Marianna & F & 17 & physics & quantum physics & string theory & sport & fishing & football\\
            Marian & M & 15 & literature & poems & n/a & n/a & n/a & n/a \\
            \end{tabular}}

    }
\end{frame}

\begin{frame}[fragile]{JSON - JavaScript Object Notation}
\begin{minted}[fontsize=\footnotesize]{js}
{
    "name": "Alice",
    "age": 17,
    "interests": [
        {
            "name": "physics",
            "disciplines": [
                "quantum physics",
                "string theory"
            ]
        },
        {
            "name": "sport",
            "disciplines": [
                "fishing",
                "football"
            ]
        }
    ]
}
\end{minted}
\end{frame}

\begin{frame}[fragile]{JSON - JavaScript Object Notation}
\begin{minted}[fontsize=\footnotesize]{js}
{
    "name": "Bob",
    "age": 15,
    "interests": [
        {
            "name": "literature",
            "genre": [
                "poems"
            ]
        }
    ]
}
\end{minted}
\end{frame}

\begin{frame}
    \frametitle{JSON - JavaScript Object Notation}
    \begin{definition}
        \emph{JavaScript Object Notation} is a lightweight text data format which is relatively easy to read for both human naked eye and computers. Although it derives from JavaScript it is a language-independent data format. JSON is built on two structures: a collection of key-item pairs and ordered list of values. JSON filenames use .json extension.
    \end{definition}
\end{frame}


\begin{frame}[fragile]{JSON Lines}
\begin{minted}[fontsize=\footnotesize]{js}
{"name":"Alice","age":17,"interests":[
    {"name":"physics","fields":["quantumphysics","stringtheory"]},
    {"name":"sport","fields":["fishing","football"]}
]}
{"name":"Bob","age":15,"interests":[
    {"name":"literature","genre":["poems"]}
]}
\end{minted}
\begin{definition}
    \emph{JSON Lines} (newline-delimited JSON) is a lightweight text data format which can be processed one record at a time. Each line consists of a JSON. JSON Lines filenames use .jl or .jsonl extensions.
\end{definition}
\end{frame}


\end{document}